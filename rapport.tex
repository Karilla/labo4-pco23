\documentclass{article}
\usepackage{graphicx}
\usepackage[margin = 2cm]{geometry}

\title{PCO Laboratoire 4 \\
\large Gestion de Ressources}
\author{Benoit Delay, Eva Ray}

\begin{document}
\maketitle

\section*{Description des fonctionnalités du logiciel}

Le programme simule le déplacement de deux locomotives, dont le comportment est exécuté par un thread. 
Chaque locomotive a un certain point de départ et suit un parcours cyclique. Les deux
parcours ont un tronçon commun, auquel une seule locomotive à la fois ne peut accéder. Ce tronçon est appelé "section partagée" ci-après. Si une
locomotive est en train d'accéder à la section partagée, l'autre s'arrête et attend son tour. 

\noindent
Chaque locomotive a une gare attirée à laquelle elle doit s'arrêter. Les deux locomotives assurent une correspondance aux passagers, 
afin qu'ils puissent changer de train si besoin. Ainsi, lorsque la première locomotive arrive en gare, elle s'arrête et attend la seconde. 
Lorsque les deux locomotives sont en gare, elles attendent encore 5 secondes avant de repartir et continuer leur parcours. 

\noindent
La priorité d'accès à la section partagée est définie comme ceci: la locomotive qui arrive en gare en dernier est prioritaire. 

\noindent
Les locomotives continuent leur parcours de manière infinie. Le seul moyen de les arrêter est d'acctionner l'arrêt d'urgence. 

\noindent
Le logiciel simule le déplacement des locomotives par des threads différents, il est donc multi-threadé. Par conséquent, il doit 
assurer une bonne gestion de la concurrence pour les ressources partagées entre plusieurs threads. En particuiler, l'accès à la section
partagée doit être géré conformément au exigences citées ci-dessus. 

\section*{Choix d'implémentation}
% Comment avez-vous abordé le problème, quels choix avez-vous fait, quelle 
% décomposition avez-vous choisie, quelles variables ont dû être protégées, ...
\subsection*{Choix du parcours}
Nous avons décidé de travailler sur la maquette \texttt{A}. L'imagination n'étant pas notre point fort, nous avons choisi de définir le parcours des
locomotives de la même manière que dans l'exemple présent dans la consigne de laboratoire. Les points de départ des locomotives, les gares et la
section critique sont mises en avant dans le schéma ci-dessous. 

\begin{center}
\includegraphics[scale = 0.6]{parcours_loco.png}
\end{center}

\subsection*{Gestion section partagée}

\subsection*{Classe Route}

\subsection*{Modularité}

\section*{Tests effectués}
\noindent

\section*{Conclusion}
% Description de chaque test, et information sur le fait qu'il ait passé ou non
\noindent


\end{document}