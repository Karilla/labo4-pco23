\documentclass{article}
\usepackage{graphicx}
\usepackage[margin = 2cm]{geometry}

\title{PCO Laboratoire 4 \\
\large Gestion de Ressources}
\author{Benoit Delay, Eva Ray}

\begin{document}
\maketitle

\section*{Description des fonctionnalités du logiciel}

Le programme simule le déplacement de deux locomotives, dont le comportment est exécuté par un thread lancé avec la méthode \texttt{LocomotiveBehavior::run()}. 
Chaque locomotive a un certain point de départ et suit un parcours cyclique. Les deux
parcours ont un tronçon commun, auquel une seule locomotive à la fois ne peut accéder. Ce tronçon est appelé "section partagée" ci-après. Si une
locomotive est en train d'accéder à la section partagée, l'autre s'arrête et attend son tour. 

Chaque locomotive a une gare attirée à laquelle elle doit s'arrêter. Les deux locomotives assurent une correspondance aux passagers, 
afin qu'ils puissent changer de train si besoin. Ainsi, lorsque la première locomotive arrive en gare, elle s'arrête et attend la seconde. 
Lorsque les deux locomotives sont en gare, elles attendent encore 5 secondes avant de repartir et continuer leur parcours. 

La priorité d'accès à la section partagée est définie comme ceci: la locomotive qui arrive en gare en dernier est prioritaire. 

Les locomotives continuent leur parcours de manière infinie. Le seul moyen de les arrêter est d'actionner l'arrêt d'urgence. 

Le logiciel simule le déplacement des locomotives par des threads différents, il est donc multi-threadé. Par conséquent, il doit 
assurer une bonne gestion de la concurrence pour les ressources partagées entre plusieurs threads. En particuiler, l'accès à la section
partagée doit être géré conformément au exigences citées ci-dessus. 

\section*{Choix d'implémentation}
% Comment avez-vous abordé le problème, quels choix avez-vous fait, quelle 
% décomposition avez-vous choisie, quelles variables ont dû être protégées, ...
\subsection*{Choix du parcours}
Nous avons décidé de travailler sur la maquette \texttt{A}. L'imagination n'étant pas notre point fort, nous avons choisi de définir le parcours des
locomotives de la même manière que dans l'exemple présent dans la consigne de laboratoire. Les points de départ des locomotives, les gares et la
section critique sont mises en avant dans le schéma ci-dessous. 

\begin{center}
\includegraphics[scale = 0.55]{parcours_loco.png}
\end{center}

\subsection*{Gestion section partagée}
Pour gérer l'accès à la section partagée, nous avons ajouté un sémaphore \texttt{waiting} et un compteur 
\\ \texttt{nbWaitingSharedSec} en attributs privés de la classe \texttt{Synchro}.
Ce sémaphore est initialisé à 0 et agit comme une barrière. Tout thread qui tente d'acquérir le sémaphore est bloqué tant qu'un
autre thread ne l'aura pas relâché. Ce sémaphore est utilisé dans la méthode \texttt{Synchro::access()}. 

Si une locomotive qui n'a pas la priorité essaie d'accéder à la section partagée, elle va acquire le sémaphore et se mettre en attente. Il faut alors
qu'une autre locomomotive release le sémaphore. Cela se fait dans la méthode \texttt{Synchro::leave()}, qui est appelée dès qu'une
locomotive quite la section partagée. Si une locomotive est en attente, on release le sémpahore \texttt{waiting} et elle peut
à son tour accéder à la section partagée.

\subsection*{Gestion attente à la gare}
Pour gérer l'attente à la gare,  nous avons ajouté un sémaphore \texttt{station} et un compteur \texttt{nbWaitingStation} en attributs
privés de la classe \texttt{Synchro}. Comme pour la section partagée, ce sémaphore est initialisé à 0 et agit comme une barrière.
La gestion de la gare se fait dans la méthode \texttt{Synchro::stopAtStation}.

Si la locomotive arrive à la gare en premier, elle doit attendre la deuxième. Pour simuler ce comportement, la première locomotive
va acquire le sémaphore \texttt{station} et va donc devoir attendre l'arrivée en gare de la seconde locomotive avant de faire sa
routine (attendre 5 secondes). La priorité de la locomotive est alors mise à 0, ce qui signifie qu'elle n'est pas prioritaire pour 
l'accès à la section partagée. Si la locomotive qui arrive en gare est la dernière à arriver, elle release le sémaphore \texttt{waiting} 
pour libérer l'autre locomotive en attente. La priorité de la locomotive est 
alors mise à 1, ce qui signifie qu'elle est prioritaire pour l'accès à la section partagée. 
Une fois les 5 secondes d'attente passées, les deux locomotives peuvent redémarrer.

\subsection*{Gestion variables partagées}
Certaines variables sont partagées entre plusieurs threads, il faut donc protéger leur accès. En particulier, les attributs
\texttt{nbWaitingSharedSec}, \texttt{nbWaitingStation} de la classe \texttt{Synchro} sont partagés entre les différents threads lancés sur 
\texttt{LocomotiveBehavior::run()}. Afin de les protéger, nous avons ajouté un sémaphore \texttt{mutex} comme attribut de la
classe \texttt{Syncrho}. Ce sémaphore est initilaisé à 1 et se comporte donc comme un mutex. Il est acquire dès qu'on doit 
modifier des variables partagées ou qu'on a besoin d'une lecture protégée, par exemple dans la condition d'un \texttt{if}.
Le sémaphore est release dès qu'on a fini notre travail. D'une manière générale, nous essayons de réduire la section critique
au minimum, afin de de ne pas bloquer des threads pour rien. Le mutex est aussi systématiquement release avant qu'un thread passe 
en attente sur les sémaphores \texttt{waiting} ou \texttt{station}, encore une fois afin de ne pas bloquer les autres threads pendant
l'attente. Une fois l'attente terminée, le mutex peut être acquire à nouveau, si besoin.

\subsection*{Classe Route}
Nous avons défini et implémenté une nouvelle classe \texttt{Route}, dont le nom est la traduction de "parcours" en anglais, qui 
représente les principales cractéristiques du parcours suivi par une certaine locomotive. Les attributs de la classe 
décrivent les points clés du parcours. Les attributs \texttt{station}, \texttt{startSharedSection} et \texttt{endSharedSection}
contiennent le numéro des points de contact de la gare, du début et de la fin de la section partagée du parcours. Dans l'attribut
\texttt{aiguillages}, on stocke le numéro et l'orientation des aiguillages permettant d'accéder et de quitter la section partagée.
L'orientation des aiguillages est propre à chaque locomotive. Tous ces attributs ont une visibilité privée, afin de garantir une
bonne encapsulation. Nous avons défini des méthodes publiques qui permettent d'attendre la détection de points de contact cités
ci-dessus et une méthode qui permet de changer la direction des aiguillages pour garantir les fonctionnalités qui peuvent être
utiles en dehors de la classe. Ces méthodes sont notamment utilisées dans \texttt{LocomotiveBehavior::run()}.

Définir une nouvelle classe est utile pour des raisons d'encapsulation des données, de lisibilité et de modularité.

\subsection*{Modularité}
Le programme est orienté pour être relativement modulable.

L'utilisation de la classe \texttt{Route} permet d'avoir différents types de parcours. Comme les directions des aiguillages
sont sockées dans un \texttt{std::Vector}, on peut imaginer un parcours qui demande le changement de direction de plus de 
deux aiguillages. Si on souhaite, par exemple, avoir plusieurs gares ou sections critiques, on peut changer les attributs qui stockent
le numéro des points de contact en \texttt{std::Vector} et adapter les méthodes correspondantes en conséquences. (Il faudrait aussi 
évidemment adapter \texttt{LocomotiveBehavior::run()} pour que le comportement des locomotives ait du sens.)

La classe \texttt{Route} nous permet aussi d'avoir une méthode \texttt{LocomotiveBehavior::run()} propre, simple et sans valeurs 
hardcodées.

Le programme a aussi été développé en pensant qu'il pourrait potentiellement y avoir plus de deux locomotives sur la maquette. Il faudrait
faire quelques ajustements mais la présence des attributs \texttt{nbWaitingSharedSec} et \texttt{nbWaitingStation} rendent le programme
extensible. Si on souhaite avoir plus deux locomotives sur la  maquette, il faut modifier la méthode \texttt{Synchro::stopAtStation} afin
que l'arrivée en gare de la dernière locomotive libère toutes les locomotives en attente, pas seulement une. Ensuite, il faudrait 
mettre en place un nouveau mécanisme de priorité d'accès à la section partagée, si on le souhaite, avec par exemple un mécanisme de file
LIFO. Dans l'état actuel des choses, à part pour la locomotive qui a la priorité, les autres locomotives accèderont à la section partagée 
selon la loi du "premier arrivé, premier servi". 

\section*{Tests effectués}
% Description de chaque test, et information sur le fait qu'il ait passé ou non
\subsection*{Test: Accès à la section partagée}

\begin{itemize}
    
\item Lorsque la locomotive A arrive en dernier à la gare, elle a la priorité pour l'accès à la section partagée.
\item Lorsque la locomotive B arrive en dernier à la gare, elle a la priorité pour l'accès à la section partagée.
\item Lorsqu'une locomotive qui n'a pas la priorité arrive à la section partagée, elle s'arrête et attend son tour.
\item Lorsqu'une locomotive quitte la section partagée, l'autre locomotive qui était en attente est débloquée et redémarre.
\item Lorsqu'une locomotive a le droit d'accès à la section partagée, les aiguillages s'orientent dans la direction définie
par le parcours de la locomotive.

\end{itemize}

Ces tests ont été faits en mettant les locomotives A et B dans chacun des rôles décrits  (en utilisant notamment le bouton pause), 
afin d'être certain que les résultats des tests ne dépendent pas de la locomotive. La priorité des locomotives accédant à la section 
partagée a pu être vérifié grâce aux logs affichés. Tous les tests produisent le résultat attendu.

\subsection*{Test: Attente à la gare}

\begin{itemize}
    
\item Lorsqu'une locomotive arrive au niveau du contact de sa gare, elle s'arrête.
\item Lorsqu'une locomotive arrive en gare en premier, elle attend que l'autre arrive avant de faire son décompte de 5 secondes.
\item Lorsque la deuxième locomotive arrive en gare, les deux locomotives attendent bien 5 secondes avant de repartir.
\item Lorsqu'une locomotive arrive en gare en premier et que la deuxième n'arrive jamais, la locomotive ne redémarre pas. 
    
\end{itemize}

Ces tests ont été faits en mettant les locomotives A et B dans chacun des rôles décrits (en utilisant notamment le bouton pause), 
afin d'être certain que les résultatsdes tests ne dépendent pas de la locomotive. Tous les tests produisent le résultat attendu.

\subsection*{Test: Arrêt d'urgence}

\begin{itemize}

\item Lorsque les deux locomotives sont en train de rouler er qu'on appuie sur le bouton d'arrêt d'urgence, les locomotives s'arrêtent comme attendu.

\item Lorsqu'une seule locomotive roule et qu'on a mis l'autre sur pause, la locomotive qui roule s'arrête et la locomotive qui était sur pause ne
repart pas quand on appuie sur start. C'est le comportement attendu.

\item Lorsqu'on appuie sur l'arrêt d'urgence alors qu'une locomotive se trouve dans la section partagée et a juste assez d'inertie pour déclencher
le capteur de sortie de secion partagée, la locomotive s'arrête et l'autre locomotive qui attend l'accès à la section partagée ne redémarre pas. 
C'est le comportement attendu.

\end{itemize}

\section*{Conclusion}
Nous avons implémenté un programme multi-threadé qui simule le déplacement des locomotives conformément au cahier des charges. Nous avons implémenté une 
gestion de la concurrence et d'accès aux ressources partagées grâce à des sémaphores, qui semblent faire leur travail convenablement. L'accès à la section 
partagée, la définition des priorités, le changement de direction des aiguillage et l'attente à la gare sont gérés correctement. Le bouton d'arrêt d'urgence 
fonctionne lui aussi comme attendu et nous permet de mettre fin à la simulation.

\end{document}